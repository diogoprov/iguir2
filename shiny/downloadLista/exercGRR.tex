\begin{defproblem}{comparMultDBC12345678}
  

  
\noindent Carregue os dados obtidos de um experimento realizado em
delineamento de blocos casualizados. Os dados estão disponíveis pelo
endereço abaixo. Os dados estão em arquivo texto com delimitador de
campos tabulação, separador decimal vírgula e primeira linha sendo o
cabeçalho. Dentro do arquivo as primeiras linhas foram comentadas e
apresentam uma descrição dos dados, como nome das variáveis,
responsáveis ou fonte.

\begin{center}
  \url{http://www.leg.ufpr.br/~walmes/data/ramalho\_arroz1.txt}
\end{center}

\begin{compactenum}
  \item Importe os dados e faça uma análise exploratória.
  \item Escreva o modelo estatístico justificado pelo delineamento e
    suas pressuposições.
  \item Ajuste o modelo aos dados e faça uma verificação sobre a
    adequação dos pressupostos. Havendo fuga dos pressupostos, tome
    medidas para adequá-los.
  \item Obtenha o quadro de análise de variância, descreva as hipóteses
    que estão sendo avaliadas e interprete os resultados.
  \item Aplique o teste de comparações múltiplas t (LSD) ao nível
    nominal de significância de 10\% e interprete os
    resultados.
  \item Apresente os resultados em um gráficos de segmentos com o
    intervalo de confiança de 90\% para a média de cada
    nível do fator estudado. Dê ao gráfico legendas apropriadas conforme
    a descrição contida no arquivo importado.
\end{compactenum}

\begin{onlysolution}
\begin{solution}
  \textcolor{A fazer.}
\end{solution}
\end{onlysolution}

\end{defproblem}
